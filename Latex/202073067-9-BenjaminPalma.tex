\documentclass[letter, 10pt]{article}
\usepackage[latin1]{inputenc}
\usepackage[spanish]{babel}
\usepackage{amsfonts}
\usepackage{amsmath}
\usepackage[dvips]{graphicx}
\usepackage{url}
\usepackage[top=3cm,bottom=3cm,left=3.5cm,right=3.5cm,footskip=1.5cm,headheight=1.5cm,headsep=.5cm,textheight=3cm]{geometry}


\begin{document}
\title{Inteligencia Artificial \\ \begin{Large}Estado del Arte: Generaci\'on de Mazmorras\end{Large}}
\author{Benjam\'in Palma}
\date{\today}
\maketitle


%--------------------No borrar esta secci\'on--------------------------------%
\section*{Evaluaci\'on}

\begin{tabular}{ll}
Resumen (5\%): & \underline{\hspace{2cm}} \\
Introducci\'on (5\%): & \underline{\hspace{2cm}} \\
Definici\'on del Problema (10\%): & \underline{\hspace{2cm}} \\
Estado del Arte (35\%): & \underline{\hspace{2cm}} \\
Modelo Matem\'atico (20\%): & \underline{\hspace{2cm}}\\
Conclusiones (20\%): & \underline{\hspace{2cm}}\\
Bibliograf\'ia (5\%): & \underline{\hspace{2cm}}\\
 & \\
\textbf{Nota Final (100\%)}:  & \underline{\hspace{2cm}}
\end{tabular}
%---------------------------------------------------------------------------%
\vspace{2cm}


\begin{abstract}
% Los juegos han estado con la humanidad desde sus inicios, como herramientas de socializaci\'on, entretenci\'on y mas importante como catalizador de creatividad y tecnolog\'ia. Hoy en d\'ia podemos ver como la industria de los juegos y las tecnol\'ogicas avanzan a pasos agigantados, notandolo con la creciente inversi\'on y ganancias en juegos, tecnol\'ogicamente en avances dentro de la computaci\'on cu\'antica, rob\'otica, IA y algoritmos. Algoritmos como la generaci\'on procedural de mazmorras que es una t\'ecnica fundamental en el dise\~no de videojuegos, permite la creaci\'on de entornos din\'amicos, personalizables y \'unicos en cada partida, al reducir la carga de dise\~no manual, estos algoritmos permiten a los desarrolladores centrarse en otros aspectos del juego, como la narrativa y la mec\'anica, fomentando la innovaci\'on de nuevo conocimiento y tecnolog\'ias. 

Los juegos han acompa\~nado a la humanidad desde sus inicios como herramientas de socializaci\'on, entretenimiento y, m\'as importante a\'un, como catalizadores de creatividad y tecnolog\'ia. Podemos observar c\'omo la industria de los juegos y la tecnolog\'ia avanzan a pasos agigantados, reflejado en la creciente inversi\'on y ganancias en este sector, as\'i como en los avances en \'areas como la computaci\'on cu\'antica, la rob\'otica, la inteligencia artificial y el desarrollo de algoritmos. T\'ecnicas como la generaci\'on procedural de mazmorras, fundamentales en el dise\~no de videojuegos, permiten la creaci\'on de entornos din\'amicos, personalizables y \'unicos en cada partida. Al reducir la carga de dise\~no manual, estos algoritmos permiten a los desarrolladores centrarse en otros aspectos del juego, como la narrativa y la mec\'anica, fomentando as\'i la innovaci\'on de nuevos conocimientos y tecnolog\'ias.
\end{abstract}

\section{Introducci\'on}
%Una explicaci\'on breve del contenido del informe, es decir, detalla: Prop\'osito, Estructura del Documento, Descripci\'on (muy breve) del Problema y Motivaci\'on.

La generaci\'on procedural se remonta a los a\~nos 70, con la llegada de los libros de {\it Dungeons \& Dragons}, cuyo objetivo era recrear una aventura \'epica mediante una narrativa interactiva. En este contexto, la imaginaci\'on humana permite crear escenarios pr\'acticamente infinitos y con interacciones complejas. Esta caracter\'istica se hizo tan popular que fue adoptada en los videojuegos mediante el g\'enero ``{\it Roguelike}'', en honor al juego ``Rogue'' \cite{roguerise}, el cual usa esta t\'ecnica en la creaci\'on de niveles tipo mazmorras, generando niveles \'unicos en cada partida. Hoy en d\'ia, los algoritmos de generaci\'on procedural buscan ofrecer esta misma libertad que caracteriza a {\it Dungeons \& Dragons}, con altos grados de personalizaci\'on, exploraci\'on, interacci\'on y unicidad.

Siguiendo la l\'inea de investigaci\'on en generaci\'on procedural, definida por Togelius et al. \cite{togelius2014procedural} como la creaci\'on de contenido por software junto a humanos o de manera aut\'onoma, este proyecto se enfoca en explorar las estrategias \'optimas para generar mazmorras, bas\'andose en los modelos de contenido procedural categorizados por Breno et al. \cite{viana2019PDG} La dificultad radica en generar un niveles coherente con barreras y llaves (las llaves abren las barreras), esto genera combinaciones no solucionables para el jugador, por lo que es una restriccion que se debe tomar en cuetna.

En las secciones posteriores, se introduce el trasfondo y la definici\'on del problema; luego, en el Estado del Arte, se exploran las t\'ecnicas actuales y los modelos matem\'aticos en el contexto de la generaci\'on procedural; finalmente, se presentan las conclusiones basadas en el an\'alisis de estrategias y su impacto en la generaci\'on de mazmorras.

\section{Definici\'on del Problema}
% Explicaci\'on del problema que se va a estudiar, en qu\'e consiste, cu\'ales son sus variables , restricciones y objetivo(s) de manera general (en palabras, no una formulaci\'on matem\'atica). Debe entenderse claramente el problema y qu\'e busca resolver.
% Explicar si existen problemas relacionados.
% Destacar, si existen, las variantes m\'as conocidas.\\
% Redactar en tercera persona, sin faltas de ortograf\'ia y referenciar correctamente sus fuentes mediante el comando \verb+\cite{ }+. Por ejemplo, para hacer referencia al art\'iculo de algoritmos h\'ibridos para problemas de satisfacci\'on 
% de restricciones~\cite{Prosser93Hybrid}.

De acuerdo con Togelius et. al \cite{togelius2014procedural} la generaci\'on procedural se refiere a alg\'un software que pueden crear contenido en conjunto con humanos, dise\~nadores o auton\'omicamente. Van Der Linden \cite{van2013procedural} describe a los niveles de mazmorras (dungeon levels) como laberintos que integran de manera interrelacionada recompensas, desaf\'ios y acertijos, los cuales el jugador debe superar para progresar en el juego.

El problema a tratar en este estudio involucra la creaci\'on de niveles de mazmorras con varios componentes clave: barreras $B$, llaves $L$, un m\'inimo de llaves requeridas $L_R$, habitaciones $H$ y el coeficiente lineal $C_L$, estos par\'ametros son dados por el usuario. Las restricciones impuestas en el dise\~no de la mazmorra exigen que todas las habitaciones $H$ est\'en conectadas de manera accesible desde la habitaci\'on inicial y que cada habitaci\'on tenga un m\'aximo de cuatro conexiones adyacentes (arriba, abajo, izquierda y derecha). En estas habitaciones pueden colocarse puertas bloqueadas $B$ que requieran llaves $L$ para abrirse; cada llave abre una puerta espec\'ifica, aunque algunas puertas pueden requerir varias llaves, y ciertas llaves pueden estar duplicadas. Para asegurar la resolubilidad del nivel, se define un m\'inimo de llaves requeridas $L_R$, y se dise\~na el algoritmo de modo que los valores de $B$, $L$, y $L_R$ aseguren accesibilidad sin bloqueos insalvables.

Para medir la linealidad o ramificaci'on de la mazmorra, se introduce el coeficiente lineal $C_L$ usado en \cite{Dumont2024}, el cual mide el n\'umero promedio de habitaciones adyacentes, sin contar la habitaci\'on por la cual uno entra influyendo en el grado de ramificaci'on y controlando la linealidad del dise\~no, con valores entre 1 y 3. Este coeficiente se calcula promediando los valores individuales de cada habitaci\'on, y, por simplicidad, no se consideran casos con bucles.

El objetivo entonces es minimizar la diferencia entre los par\'ametros dados por el usuario sobre los valores de $H$, $B$, $L$, $L_R$ y $C_L$ los generados por el algoritmo, para que el dise\~no final de la mazmorra respete los par'ametros y criterios especificados en la mayor medida posible.

\section{Estado del Arte}
% La informaci\'on que describen en este punto se basa en los estudios realizados con antelaci\'on respecto al tema.
% Lo m\'as importante que se ha hecho hasta ahora con relaci\'on al problema. Deber\'ia responder preguntas como las siguientes:
% ?`cu\'ando surge?, ?`qu\'e m\'etodos se han usado para resolverlo?, ?`cu\'ales son los mejores algoritmos que se han creado hasta
% la fecha?, ?`qu\'e representaciones han tenido los mejores resultados?, ?`cu\'al es la tendencia actual para resolver el problema?, tipos de movimientos, heur\'isticas, m\'etodos completos, tendencias, etc... Puede incluir gr\'aficos comparativos o explicativos.

El dise\~no de entornos coherentes y visualmente atractivos es fundamental en la generaci\'on de mazmorras. Breno et al. \cite{viana2019PDG} que proponen una taxonom\'ia para la generaci\'on procedural de mazmorras, clasific\'andolas en cuatro categor\'ias: tipo de mazmorra, elementos y mec\'anicas de juego, generaci\'on de contenido procedimental y estrategias de soluci\'on. Seg\'un la taxonom\'ia los autores concluyen que creaci\'on de mazmorras condicionadas por barreras y llaves es un \'area subestudiada, mostrando 4 de 26 documentos revisados entre 2011 y 2019 en CM Digital Library, IEEE Xplore y Scopus abordaron este tema espec\'ifico. En cuanto a las estrategias de soluci\'on, los algoritmos gen\'eticos y constructivos son los m\'as comunes para la generaci\'on de mazmorras. Sin embargo, en el contexto espec\'ifico de problemas con barreras, no se ven estos m\'etodos, sino enfoques alternativos, como gram\'atica generativa, programaci\'on gen\'etica y programaci\'on de conjuntos de respuestas ({\it Answer Set Programming} ASP). Estudios recientes (2021-2024) \cite{Dumont2024,PEREIRA2021115009} han investigado algoritmos evolutivos, tambi\'en se aprecia un marco de trabajo de dos iteraciones en \cite{Dumont2024,10.1145/3337722.3341847}, donde se basan en generar primero la mazmorra y luego la l\'ogica de llaves y barreras. En el estado del arte actual (2023-2024), se han incorporado tecnolog\'ias como Modelos Largos de Lenguaje (LLM) para resolver problema, inspirados fuera de nuestro contexto en \cite{holygrail2.0} donde usan a los modelos como interfaces para transformar problemas de lenguaje natural en c\'odigo para solvers de satisfacci\'on y optimizaci\'on (ejemplo, MiniZinc). Y luego paper como \cite{word2world} donde usan a estos modelos como n\'ucleo para crear niveles con restricciones con marcos de trabajo iterativos, donde la generaci\'on, reparaci\'on y evaluaci\'on van convergiendo a una soluci\'on. 

Considerando la experiencia previa en la aplicaci\'on de gram\'atica generativa en este contexto, se presenta una oportunidad prometedora para explorar el potencial de los Modelos Largos de Lenguaje (LLM) en la resoluci\'on de nuestro problema. Nuestro enfoque innovador consiste en dise\~nar un modelo representable en texto, que pueda ser procesado mediante un gran modelo del lenguaje. Esta aproximaci\'on permite aprovechar las capacidades de comprensi\'on y generaci\'on de lenguaje natural de los LLM. En este punto realizaremos dos pasos como mencionan en \cite{Dumont2024,10.1145/3337722.3341847}, primero generaremos la mazmorra y validaremos su correcta generaci\'on, en el segundo paso ubicaremos las llaves y barreras. Posteriormente, evaluaremos la efectividad de la soluci\'on generado mediante algoritmos especializados basados en AStar que verifiquen el cumplimiento de las restricciones y requisitos espec\'ificos de nuestro problema. Este enfoque integrado combina la potencia de los LLM con la precisi\'on de los algoritmos de evaluaci\'on, lo que nos permite abordar el problema desde una perspectiva m\'as hol\'istica y eficiente. 

Nuestro m\'etodo se puede describir en las siguientes etapas:
\begin{enumerate}
    \item Dise\~no del modelo representable en texto.
    \item Generaci\'on de una soluci\'on de esquema de mazmorra mediante un gran modelo del lenguaje (LLM).
    \item Evaluaci\'on de la soluci\'on mediante algoritmos de verificaci\'on de restricciones.
    \item Instanciar llaves y barreras dentro de la Dungeon
    \item Validar correcta instanciacion
    \item Soluci\'on, sino iterar hasta 5 veces desde 4 en adelante con nueva informaci\'on (heur\'isticas), sino volver a 2 y cambiar el esquema de la mazmorra.
\end{enumerate}
\section{Modelo Matem\'atico}
% Uno o m\'as modelos matem\'aticos para el problema, idealmente indicando el espacio de b\'usqueda para cada uno. Cada modelo debe estar correctamente referenciado, adem\'as no debe ser una imagen extraida. Tambi\'en deben explicarse en detalle cada una de las partes, mostrando claramente la funci\'on a maximizar/minimizar, variables y restricciones. Tanto las f\'ormulas como las explicaciones deben ser consistentes.
\subsection*{Modelamiento como problema de restricciones y optimizaci\'on cl\'asico (COP)}

\subsection*{Funci\'on Objetivo}
La funci\'on objetivo busca minimizar la diferencia absoluta entre los valores ideales y los alcanzados de habitaciones, llaves, barreras, linealidad y llaves necesarias:
\[
f(x) = 2 (\Delta_{H}+\Delta_{L}+\Delta_{B}+\Delta_{C_L}) + \Delta_{L_R}
\]
donde:
\begin{itemize}
    \item $\Delta_H$ es la diferencia absoluta en el n\'umero de habitaciones pedidas ($\widehat{H}$) y instanciadas ($H$).
    \item $\Delta_L$ es la diferencia en el n\'umero de llaves pedidas ($\widehat{L}$) y instanciadas ($L$).
    \item $\Delta_B$ es la diferencia en el n\'umero de barreras pedidas ($\widehat{B}$) y instanciadas ($B$).
    \item $\Delta_{C_L}$ es la diferencia en el coeficiente de linealidad pedidos ($\widehat{C_L}$) y instanciados ($C_L$).
    \item $\Delta_{L_R}$ es la diferencia en el n\'umero de llaves necesarias pedidas ($\widehat{L_R}$) y instanciadas ($L_R$).
\end{itemize}

En un inicio las instancias y lo pedido ser\'an la misma, pero si no llegara a existir soluci\'on el algoritmo empezar\'ia a variar primero las llaves necesarias pedidas y luego lo dem\'as en orden de hacer el nivel soluci\'onale

\subsubsection*{Variables}
\begin{itemize}
    \item $x_{ij}$: Binaria $x_{ij} \in \{0,1\}$, indica si existe una conexi\'on entre la habitaci\'on $i$ y la habitaci\'on $j$, donde $\{i \in \mathbb{N} \ |\ 0\leq i \leq H,\ H \in \mathbb{N}\}$ y $\{j \in \mathbb{N} \ |\ 0\leq j \leq H,\ H \in \mathbb{N} \land j \neq i\}$ .
    
    \item $k_b$: N\'umero de llaves requeridos para abrir la barrera $b \in \{b \in \mathbb{N} \ |\ 0\leq b \leq B,\ B \in \mathbb{N}\}$, $k_b \in \mathbb{N}$.
    
    \item $b_{l}$: Barreras n-\'ecima y su asignaci\'on a las llaves l-\'esima (uno a uno). $b_{l} \in \{b_{l} \in \mathbb{N} \ |\ 0\leq b_{l} \leq B,\ B \in \mathbb{N}\}$ y $l \in \{l \in \mathbb{N} \ |\ 0\leq l \leq L,\ L \in \mathbb{N}\}$
    
    \item $C_L$: Coeficiente de linealidad (medido en funci\'on de la linealidad de la secuencia de conexi\'on entre habitaciones). $C_L \in [1,3]$
    
    \item $L_R$: N\'umero de llaves necesarias para completar el nivel. $L_R \in \{L_R \in \mathbb{N} \ |\ B\leq L_R \leq L,\ B,L\in \mathbb{N}\}$
\end{itemize}



\subsubsection*{Restricciones}
\begin{enumerate}
    \item \textbf{Conectividad entre habitaciones}:
    \begin{itemize}
        \item \textbf{Conexi\'on m\'inima}: Todas las habitaciones deben estar conectadas al menos a otra habitaci\'on:
        \[
        \sum_{j} x_{ij} \geq 1, \quad \forall i
        \]
        \item \textbf{Conexi\'on m\'axima}: Cada habitaci\'on puede tener un m\'aximo de 4 conexiones (arriba, derecha, abajo, izquierda):
        \[
        \sum_{j} x_{ij} \leq 4, \quad \forall i
        \]
    \end{itemize}
    
    \item \textbf{Asignaci\'on de llaves y barreras}:
    \begin{itemize}
        \item Cada llave tiene una barrera asignada, pero una barrera puede requerir varias llaves:
        \[
        k_b \geq 1, \quad \forall b
        \]
    \end{itemize}
    
    \item \textbf{Compleci\'on del nivel}:
    \begin{itemize}
        \item El n\'umero de llaves asignadas debe ser suficiente para pasar todas las barreras y llegar al final del nivel:
        \[
        \sum_{b} k_b \leq l_N
        \]
    \end{itemize}
\end{enumerate}

\subsection*{Soluci\'on orientado a un algoritmo de 2 pasos evolutivo}

Nuestra propuesta actual para la generaci\'on de mazmorras se basa en una representaci\'on mediante una estructura de \'arbol, como se describe en \cite{Dumont2024}. En esta representaci\'on, cada nodo del \'arbol corresponde a una habitaci\'on, y puede tener un m\'aximo de tres hijos, que representan las habitaciones adyacentes hacia el oeste, sur y este, respectivamente. El nodo padre, por su parte, se sit\'ua siempre al norte del nodo actual. Este dise\~no permite una orientaci\'on espacial flexible de las habitaciones, evitando configuraciones r\'igidas y uniformes que limiten la variedad de mazmorras generadas. Como resultado, la mazmorra puede crecer homog\'eneamente en todas las direcciones, adapt\'andose mejor a los objetivos establecidos.

La representaci\'on en forma de \'arbol ofrece una flexibilidad estructural que resulta particularmente \'util para aplicar operadores evolutivos como mutaciones y cruces. Por ejemplo, las mutaciones pueden consistir en la adici\'on o eliminaci\'on de hojas en el \'arbol, lo que equivale a agregar o quitar habitaciones en la mazmorra. Esta capacidad de modificar din\'amicamente la estructura permite ajustar el dise\~no de la mazmorra para que cumpla con requisitos espec\'ificos, como el n\'umero de habitaciones o el coeficiente lineal, que es un elemento clave en la evaluaci\'on de su dise\~no. El coeficiente lineal se calcula como el promedio de hijos de los nodo que no sean hojas.

Se inicializa la poblaci\'on inicial mediante una distribuci\'on normal para generar candidatos r\'apidamente aceptables pero diversos, unicamente enfocado en la cantidad de habitaciones, no se toma en cuenta el coeficiente lineal, ya que con las iteraciones ira convergiendo.

El proceso de cruzamiento entre mazmorras se pensaba realizar seleccionando sub\'arboles de cada progenitor, contando con un par\'ametro que es max\_depth que permite calcular la profundidad de un sub \'arbol contando desde la hoja hacia arriba, para posteriormente realizar un intercambio entre ellos. Este mecanismo genera dos nuevos individuos, o hijos, que combinan caracter\'isticas estructurales de ambos padres. El objetivo principal del cruzamiento es introducir diversidad en la poblaci\'on de mazmorras, promoviendo dise\~nos novedosos que se acerquen a los par\'ametros deseados, como la cantidad de habitaciones y el coeficiente lineal ideal. El gran problema es que por tiempo no se logro terminar la implementaci\'on, por lo que optamos por grandes tazas mutaciones (y variables)

Una vez generado un candidato de mazmorra que cumple con los criterios b\'asicos, se procede a su refinamiento mediante la colocaci\'on aleatoria de barreras y llaves dentro de las habitaciones. Este paso inicial establece la distribuci\'on de elementos interactivos en el entorno. A continuaci\'on, se eval\'ua la viabilidad del dise\~no mediante un recorrido desde la ra\'iz del \'arbol, utilizando un algoritmo Dijkstra para determinar si es posible recolectar las llaves necesarias para abrir cada barrera en el camino. Si el algoritmo confirma que todas las barreras pueden ser desbloqueadas, se establece la salida de la mazmorra justo detr\'as de la \'ultima barrera, garantizando que el dise\~no final sea tanto funcional como desafiante para el jugador.

Esta estrategia no solo asegura que las mazmorras generadas sean solucionables, sino que tambi\'en permite iterar y ajustar el dise\~no para cumplir con par\'ametros espec\'ificos de dificultad, complejidad y distribuci\'on espacial. La combinaci\'on de una representaci\'on en forma de \'arbol con operadores evolutivos y t\'ecnicas de evaluaci\'on basadas en algoritmos de b\'usqueda asegura un equilibrio entre la aleatoriedad inherente al proceso y la capacidad de cumplir con los objetivos de dise\~no establecidos.

\subsection*{Instancias de Arboles Generados}
parametros extra:
\begin{itemize}
    \item steps = 30; pasos evolutivos.
    \item popSize = 50; Tama\~no de la poblaci\'on.
    \item stdDev = 2.0; Desviaci\'on est\'andar base aplicada (crece proporcional a la instancia, ya que es proporcional a las habitaciones).
    \item minMut = 5; m\'inimo de mutaciones.
    \item maxMut = 15; m\'aximo de mutaciones.
    \item crossoverProb = 0; Probabilidad de crossover, no se logro implementar correctamente.
    \item mutationProb = 1; Probabilidad de mutaci\'on, se aumento para remplazar la falta de crossover
    \item searchTry = 1000; iteraciones por \'arbol para buscar una soluci\'on mediante Dijkstra.
\end{itemize}

\begin{table}[h!]
\centering
\begin{tabular}{|c|c|c|c|c|}
\hline
\textbf{Instancia} & \textbf{Nodos} & \textbf{Coef. Real} & \textbf{Coef. Objetivo} & \textbf{Evaluaci\'on} \\
\hline
Instancia 1 & 15 & 2.00 & 2.00 & 0 \\
Instancia 2 & 20 & 1.19 & 1.00 & 0.1875 \\
Instancia 3 & 20 & 1.73 & 2.00 & 0.2727 \\
Instancia 4 & 25 & 1.14 & 1.00 & 0.1429 \\
Instancia 5 & 30 & 2.00 & 2.00 & 0 \\
Instancia 6 & 30 & 1.53 & 1.50 & 0.0263 \\
Instancia 7 & 102 & 1.51 & 1.50 & 0 \\
Instancia 8 & 500 & 1.50 & 1.50 & 0 \\
\hline
\end{tabular}
\caption{Resultados de la evaluaci\'on de instancias de mazmorras generadas.}
\end{table}

\subsection*{An\'alisis}

\begin{itemize}
    \item \textbf{Instancia 1}: 15/15 nodos, coeficiente 2.00, que es el valor ideal. El \'arbol es completamente resolvible, con una evaluaci\'on de 0, lo que indica una perfecta coincidencia con los objetivos.
    
    \item \textbf{Instancia 2}: 20/20 nodos, coeficiente 1.19, con una ligera discrepancia respecto al coeficiente objetivo de 1.00. La evaluaci\'on fue 0.1875, indicando una buena aproximaci\'on al objetivo.
    
    \item \textbf{Instancia 3}: 20/20 nodos, coeficiente 1.73. Similar a la Instancia 2, con una evaluaci\'on de 0.2727, lo que sugiere que el algoritmo ha logrado generar una mazmorra razonablemente cercana al coeficiente objetivo de 2.00.
    
    \item \textbf{Instancia 4}: 25/25 nodos, coeficiente 1.14, con una evaluaci\'on de 0.1429. El coeficiente real est\'a algo lejos del objetivo de 1.00, pero a\'un as\'i, la mazmorra es resolvible.
    
    \item \textbf{Instancia 5}: 30/30 nodos, coeficiente 2.00, lo que coincide exactamente con el coeficiente objetivo de 2.00, con una evaluaci\'on de 0, lo que indica una buena resoluci\'on del problema.
    
    \item \textbf{Instancia 6}: 30/30 nodos, coeficiente 1.53, con una evaluaci\'on de 0.0263. La evaluaci\'on muestra una ligera desviaci\'on respecto al objetivo, pero la mazmorra sigue siendo resolvible.
    
    \item \textbf{Instancia 7}: 100/100 nodos, coeficiente 1.51, con un coeficiente objetivo de 1.50, alcanzando una evaluaci\'on de 0, indicando que la mazmorra es completamente resolvible.
    
    \item \textbf{Instancia 8}: 500/- nodos, coeficiente 1.50, con un coeficiente objetivo de 1.50, pero no se logr\'o resolver esta instancia en particular, es decir la segunda fase de barreras y llaves.
\end{itemize}

De las 8 instancias generadas, 7 fueron resueltas exitosamente, con un desempe\~no cercano o alcanzando los coeficientes objetivo. La instancia 8, sin embargo, no logr\'o ser resuelta debido a la complejidad y al alto n\'umero de nodos requeridos. Esto se debe, en parte, a que la l\'ogica de posicionamiento de barreras y llaves es aleatoria, por lo que, a mayor n\'umero de nodos, mayor entrop\'ia y, por ende, m\'as estados no solucionables.


\section{Conclusiones}
% Conclusiones RELEVANTES del estudio realizado. Deber\'ia responder a las preguntas: ?`todas las t\'ecnicas resuelven el mismo problema o hay algunas diferencias?, ?`En qu\'e se parecen o difieren las t\'ecnicas en el contexto del problema?, ?`qu\'e limitaciones tienen?, ?`qu\'e t\'ecnicas o estrategias son las m\'as prometedoras?, ?`existe trabajo futuro por realizar?, ?`qu\'e ideas usted propone como lineamientos para continuar con investigaciones futuras?

La propuesta presentada combina una representaci\'on de mazmorras mediante \'arboles con operadores evolutivos y una evaluaci\'on basada en algoritmos de b\'usqueda, logrando dise\~nos flexibles y funcionales. La estructura arb\'orea facilita la generaci\'on y modificaci\'on de mazmorras, permitiendo ajustes din\'amicos para cumplir con requisitos espec\'ificos, como el n\'umero de habitaciones y el coeficiente lineal, que asegura equilibrio entre jugabilidad y complejidad.

En las pruebas realizadas, 7 de las 8 instancias generadas fueron completamente resolvibles, alcanzando un desempe\~no cercano o coincidente con los coeficientes objetivo. La instancia 8, con un mayor n\'umero de nodos, no fue resoluble debido a la complejidad a\~nadida. Como posibles mejoras, se podr\'ia evaluar la implementaci\'on de una heur\'istica para la colocaci\'on de barreras y llaves, adem\'as de sustituir Dijkstra por el algoritmo A* para resolver las instancias de manera m\'as eficiente. Finalmente, se sugiere optimizar la funci\'on de crossover para permitir una mayor exploraci\'on del espacio de soluciones.

\section{Bibliograf\'ia}
% Indicando toda la informaci\'on necesaria de acuerdo al tipo de documento revisado. Todas las referencias deben ser citadas en el documento.
\bibliographystyle{plain}
\bibliography{Referencias}

\end{document} 